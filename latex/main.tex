\documentclass[conference]{IEEEtran}
\IEEEoverridecommandlockouts

% -------------------- Packages --------------------
\usepackage{cite}
\usepackage{amsmath,amssymb,amsfonts}
\usepackage{graphicx}
\usepackage{textcomp}
\usepackage{xcolor}
\usepackage{booktabs}
\usepackage{multirow}
\usepackage{array}
\usepackage{url}
\usepackage{hyperref}
\usepackage{float}

\hypersetup{
    colorlinks=true,
    linkcolor=blue,
    citecolor=blue,
    urlcolor=blue
}

% Figures live under latex/figures (copied from outputs/figures)
\graphicspath{{figures/}}

% -------------------- Title --------------------
\title{Predicting Student Exam Scores Using Tabular Machine Learning\\
\thanks{University Kaggle Competition Report (Playground Series S6E1)}
}

% -------------------- Authors --------------------
\author{
\IEEEauthorblockN{Student 1}
\IEEEauthorblockA{\textit{Department} \\
\textit{University Name}\\
City, Country \\
email@domain.com}
\and
\IEEEauthorblockN{Tina Kovačević}
\IEEEauthorblockA{\textit{FEUP} \\
\textit{University of Porto}\\
Porto, Portugal \\
up202501724@up.pt}
\and
\IEEEauthorblockN{Zakariea Sharfeddine}
\IEEEauthorblockA{\textit{FEUP} \\
\textit{University of Porto}\\
Porto, Portugal \\
up202501730@up.pt}
}

\begin{document}
\maketitle

% -------------------- Abstract --------------------
\begin{abstract}
This project addresses the Kaggle Playground Series S6E1 competition: predicting student exam scores from demographic, academic, and behavioral features.
Using a dataset of 630,000 training samples with 12 original features, we developed a comprehensive machine learning pipeline including feature engineering, Bayesian hyperparameter optimization, ensemble methods, and deep learning.
We evaluated traditional models (Ridge, Random Forest), gradient boosting methods, e.g. LightGBM, XGBoost, CatBoost), and a novel SE-ResNet neural network architecture combining entity embeddings, residual connections, and Squeeze-and-Excitation attention.
Our best solution, the SE-ResNet with data augmentation, achieved a cross-validation RMSE of \textbf{8.6047}, outperforming the best gradient boosting ensemble (8.7567) by 1.74\%.
SHAP analysis revealed that study hours and an engineered domain formula were the strongest predictors.
We discuss model interpretability, the advantages of neural networks for tabular data, and trade-offs between different approaches.
\end{abstract}

% -------------------- Keywords --------------------
\begin{IEEEkeywords}
tabular deep learning, gradient boosting, squeeze-and-excitation, entity embeddings, SHAP, ensemble learning
\end{IEEEkeywords}

% -------------------- Sections --------------------
\section{Introduction}
Academic performance prediction is a significant machine learning application with implications for personalized learning and early intervention systems.
In this work, we address the Kaggle Playground Series S6E1 competition, where the objective is to predict student exam scores from behavioral and contextual features including study hours, class attendance, sleep patterns, and study methodology.

The dataset comprises 630,000 training samples and 270,000 test samples, generated synthetically from a deep learning model trained on real exam score data.
This provides a controlled environment to compare various machine learning approaches.

The main contributions of this paper are:
\begin{itemize}
    \item A comprehensive exploratory analysis revealing that \texttt{study\_hours} (r=0.76) is the strongest predictor of exam scores.
    \item Feature engineering expanding 12 original features to 44 derived features including interactions, target encoding, and domain-specific formulas.
    \item Comparison of 7 regression models from Ridge to gradient boosting ensembles.
    \item Achievement of 8.7568 CV RMSE using an optimized weighted ensemble of CatBoost, LightGBM, and XGBoost.
    \item Analysis of feature importance and prediction error patterns.
\end{itemize}

\section{Technical Background}
This task is a supervised regression problem. Given an input feature vector $x \in \mathbb{R}^d$, we aim to predict a continuous target value $y$, representing the exam score.

\subsection{Root Mean Squared Error (RMSE)}
The primary evaluation metric is Root Mean Squared Error:
\begin{equation}
\mathrm{RMSE} = \sqrt{\frac{1}{N}\sum_{i=1}^{N}(y_i - \hat{y}_i)^2}
\end{equation}
Lower RMSE indicates better predictive performance.

\subsection{Baseline Models}
We consider standard regression baselines such as Linear Regression and Ridge Regression, which are simple and interpretable.

\subsection{Gradient Boosted Decision Trees}
We also evaluate gradient boosting models (e.g., XGBoost, LightGBM, CatBoost), which often perform strongly on structured/tabular datasets by combining many weak decision tree learners into a robust model.

\section{Dataset Description and Preprocessing}
We use the dataset provided in the Kaggle Playground Series S6E1 benchmark, which
consists of synthetically generated student records derived from a model trained
on real student performance data. The training split contains 630{,}000 samples
with the target variable \texttt{exam\_score}, while the test split contains
270{,}000 samples used for final prediction. Figure~\ref{fig:target_distribution}
visualizes the distribution of \texttt{exam\_score}, which is approximately
symmetric and concentrated around mid-range values.

\begin{figure}[t]
\centering
\includegraphics[width=0.98\linewidth]{target_distribution.png}
\caption{Distribution of the target variable \texttt{exam\_score} (histogram and box plot).}
\label{fig:target_distribution}
\end{figure}
\subsection{Feature Overview}
Table~\ref{tab:feature_overview}  summarizes the numerical and categorical input
variables.

\begin{table}[h]
\centering
\renewcommand{\arraystretch}{1.1}
\caption{Overview of input features.}
\label{tab:feature_overview}
\begin{tabular}{l l l}
\hline
\textbf{Type} & \textbf{Feature} & \textbf{Range / Categories} \\
\hline
\multirow{4}{*}{Numerical} 
  & \texttt{age}               & 17--24 \\
  & \texttt{study\_hours}      & 0.08--7.91 \\
  & \texttt{class\_attendance} & 40.6--99.4\% \\
  & \texttt{sleep\_hours}      & 4.1--9.9 \\
\hline
\multirow{7}{*}{Categorical}
  & \texttt{gender}            & male, female \\
  & \texttt{course}            & B.Tech, B.Sc, etc. \\
  & \texttt{internet\_access}  & yes, no \\
  & \texttt{sleep\_quality}    & good, average, poor \\
  & \texttt{study\_method}     & category labels \\
  & \texttt{facility\_rating}  & category labels \\
  & \texttt{exam\_difficulty}  & category labels \\
\hline
\end{tabular}
\end{table}

\subsection{Preprocessing}
We verified that the dataset contains no missing values in
either the training or test split. Categorical variables were
converted into numerical representations using label encoding, 
and we additionally applied frequency encoding to capture 
category prevalence. To incorporate target-dependent information
while avoiding leakage, we implemented cross-validation-safe
target encoding, where category statistics are computed using
only the training folds and then applied to the corresponding
validation fold. All preprocessing steps were applied consistently 
across training and test data to ensure compatibility with
both tree-based models and neural architectures.

\subsection{Exploratory Data Analysis}
Exploratory analysis was conducted to assess relationships between numerical
features and the target variable. As shown in
Figure~\ref{fig:corr_matrix}, \texttt{study\_hours} exhibits the strongest linear
association with \texttt{exam\_score} ($r=0.762$), followed by
\texttt{class\_attendance} ($r=0.361$) and \texttt{sleep\_hours} ($r=0.167$).
In contrast, \texttt{age} shows negligible correlation ($r=0.011$).

These findings are consistent with the scatter plots in
Figure~\ref{fig:scatter_num_target}, which indicate a clear positive relationship
between study time and exam performance, weaker but positive effects for
attendance and sleep duration, and no meaningful trend for age. Low pairwise
correlations among numerical features suggest limited multicollinearity.
Overall, the analysis identifies \texttt{study\_hours},
\texttt{class\_attendance}, and \texttt{sleep\_hours} as the most informative
numerical predictors, motivating the use of interaction features and nonlinear
models in later sections.

\begin{figure}[t]
\centering
\includegraphics[width=0.98\linewidth]{correlation_matrix.png}
\caption{Pearson correlation matrix of numerical features and the target \texttt{exam\_score}.}
\label{fig:corr_matrix}
\end{figure}

\begin{figure}[t]
\centering
\includegraphics[width=0.95\linewidth]{scatter_numerical_vs_target.png}
\caption{Scatter plots of numerical features versus \texttt{exam\_score} with linear trend lines (sampled for visualization).}
\label{fig:scatter_num_target}
\end{figure}

\section{Methodology}
Our modeling pipeline progresses from tabular baselines to tuned gradient boosting models, ensembling, and a neural architecture for tabular data.


\subsection{Validation Strategy}
All experiments use 5-fold cross-validation with shuffling (random seed 42) to obtain reliable performance estimates and reduce overfitting.
We reuse the same fold splits across all models to ensure fair comparison, and we store out-of-fold (OOF) predictions for ensemble construction and weight optimization.


\subsection{Feature Engineering}
Starting from 12 original features, we construct a set of 44 features to capture nonlinearities, interactions, and category-specific effects.
For numerical variables, we include pairwise interaction terms (e.g., \texttt{study\_hours} $\times$ \texttt{class\_attendance}), ratio features (e.g., attendance-per-hour), and polynomial expansions (squared and cubic terms) for the most relevant predictors.
We additionally introduce threshold-based indicator variables (e.g., low sleep and high study intensity) and a composite score based on domain intuition:
\begin{equation}
S = 6.0h + 0.35a + 1.5s,
\label{eq:domain_score}
\end{equation} where $h=\texttt{study\_hours}$, $a=\texttt{class\_attendance}$, and $s=\texttt{sleep\_hours}$.
Categorical variables are encoded using label encoding and augmented with CV-safe target encoding and frequency encoding to expose category-specific signal while avoiding leakage.


\subsection{Models Evaluated}
We evaluate four groups of regressors: classical baselines, gradient-boosted decision trees, ensemble methods, and a neural model for tabular data.

\subsubsection{Baselines}
As classical baselines, we train Ridge regression ($\alpha=10.0$) on standardized features and a Random Forest regressor with 100 trees (max depth 15, minimum leaf size 10).

\subsubsection{Gradient-Boosted Decision Trees}
We benchmark three gradient-boosted decision tree implementations: LightGBM, XGBoost, and CatBoost.
Early stopping is applied where supported to mitigate overfitting, and final model settings are selected either from baseline configurations or via the tuning procedure described in Section~\ref{sec:hpo}.

\subsubsection{Ensemble Methods}
We evaluate three ensemble strategies built from the gradient-boosted models.
These include a simple mean blend, a weighted blend with weights optimized on out-of-fold predictions using Nelder--Mead, and stacking with a Ridge regression meta-learner.

\subsubsection{Neural Tabular Model}
Finally, we evaluate SE-ResNet, a neural architecture for mixed categorical--numerical data that combines entity embeddings with residual blocks and Squeeze-and-Excitation attention.


\subsection{SE-ResNet Architecture}
To model nonlinear interactions in mixed categorical--numerical inputs, we employ a tabular neural network based on residual learning with Squeeze-and-Excitation (SE) attention.
Numerical features are standardized and augmented with selected engineered transformations, while each categorical feature is mapped to a learnable embedding.
All inputs are concatenated and projected to a fixed hidden width, followed by multiple residual blocks with channel-wise attention.
The model outputs a single scalar prediction for \texttt{exam\_score}.
In addition to the composite score $S$ defined in Eq.~\eqref{eq:domain_score}, we include a calibrated variant as an additional neural input:
\begin{equation}
\begin{aligned}
\tilde{S} &= 5.91h + 0.35a + 1.42s + 4.78,
\end{aligned}
\label{eq:domain_score_nn}
\end{equation}
where $h=\texttt{study\_hours}$, $a=\texttt{class\_attendance}$, and $s=\texttt{sleep\_hours}$. Table~\ref{tab:seresnet_config} summarizes the SE-ResNet architecture and training configuration used in our experiments.

\begin{table}[t]
\centering
\renewcommand{\arraystretch}{1.1}
\caption{SE-ResNet configuration used in our experiments.}
\label{tab:seresnet_config}
\begin{tabular}{l l}
\hline
\textbf{Component} & \textbf{Setting} \\
\hline
Categorical representation & entity embeddings, $d=8$ per feature \\
Input projection & 256 hidden dimensions \\
Residual blocks & 3 blocks \\
Block structure & LayerNorm $\rightarrow$ Linear $\rightarrow$ ReLU $\rightarrow$ Dropout \\
Dropout & 0.11 \\
SE module & reduction ratio $r=4$ \\
Prediction head & 256 $\rightarrow$ 64 $\rightarrow$ 16 $\rightarrow$ 1 \\
Optimizer & AdamW (lr=$10^{-3}$, weight decay=$10^{-4}$) \\
LR scheduler & ReduceLROnPlateau (factor 0.5, patience 10) \\
Early stopping & patience 20 epochs \\
Batch size & 256 (train), 1024 (validation) \\
\hline
\end{tabular}
\end{table}




\subsection{Hyperparameter Tuning}
\label{sec:hpo}
We tuned the gradient boosting models using Bayesian optimization with Optuna~\cite{optuna}.
For each model, we ran 50 trials and evaluated each trial using 5-fold cross-validation, minimizing the RMSE.
The hyperparameter ranges explored for LightGBM, XGBoost, and CatBoost are summarized in Table~\ref{tab:search_space}.
Unless stated otherwise, parameters were sampled uniformly within their ranges, while \texttt{learning\_rate}, \texttt{reg\_alpha}, and \texttt{reg\_lambda} were sampled log-uniformly to better cover multiple orders of magnitude.
The best-performing configurations selected by Optuna for each model are reported in Table~\ref{tab:tuned_params}.

\begin{table}[h]
\centering
\renewcommand{\arraystretch}{1.1}
\caption{Optuna search spaces used for Bayesian hyperparameter optimization.}
\label{tab:search_space}
\begin{tabular}{l l l}
\hline
\textbf{Model} & \textbf{Hyperparameter} & \textbf{Range} \\
\hline
\multirow{7}{*}{LightGBM}
  & \texttt{num\_leaves}         & $[20, 100]$ \\
  & \texttt{learning\_rate}      & $[0.01, 0.1]$ \\
  & \texttt{feature\_fraction}   & $[0.6, 1.0]$ \\
  & \texttt{bagging\_fraction}   & $[0.6, 1.0]$ \\
  & \texttt{min\_child\_samples} & $[5, 50]$ \\
  & \texttt{reg\_alpha}          & $[10^{-3}, 10]$ \\
  & \texttt{reg\_lambda}         & $[10^{-3}, 10]$ \\
\hline
\multirow{6}{*}{XGBoost}
  & \texttt{max\_depth}          & $[3, 10]$ \\
  & \texttt{learning\_rate}      & $[0.01, 0.1]$ \\
  & \texttt{subsample}           & $[0.6, 1.0]$ \\
  & \texttt{colsample\_bytree}   & $[0.6, 1.0]$ \\
  & \texttt{reg\_alpha}          & $[10^{-3}, 10]$ \\
  & \texttt{reg\_lambda}         & $[10^{-3}, 10]$ \\
\hline
\multirow{4}{*}{CatBoost}
  & \texttt{depth}               & $[4, 10]$ \\
  & \texttt{learning\_rate}      & $[0.01, 0.1]$ \\
  & \texttt{l2\_leaf\_reg}       & $[1, 10]$ \\
  & \texttt{bagging\_temperature}& $[0.0, 1.0]$ \\
\hline
\end{tabular}
\end{table}


\section{Experiments and Results}

\subsection{Experimental Setup}
All experiments used identical preprocessing, feature engineering, and 5-fold cross-validation splits.
Performance is measured by mean CV RMSE (Root Mean Squared Error), with standard deviation indicating stability across folds.

\subsection{Model Comparison}
Table~\ref{tab:results} summarizes the performance of each method, sorted by CV RMSE.

\begin{table}[H]
\centering
\caption{Model Performance Comparison (Lower RMSE is Better)}
\label{tab:results}
\begin{tabular}{l c c}
\toprule
\textbf{Model} & \textbf{CV RMSE} & \textbf{Std} \\
\midrule
\textbf{SE-ResNet (Neural Network)} & \textbf{8.6047} & $\pm$0.015 \\
\midrule
Stacking (Ridge Meta) & 8.7567 & -- \\
Ensemble (Optimized) & 8.7568 & -- \\
CatBoost & 8.7618 & $\pm$0.0101 \\
LightGBM & 8.7724 & $\pm$0.0089 \\
Ensemble (Simple) & 8.7730 & -- \\
Ridge Regression & 8.8887 & $\pm$0.0105 \\
Random Forest & 8.9095 & $\pm$0.0109 \\
XGBoost & 8.9271 & $\pm$0.1928 \\
\bottomrule
\end{tabular}
\end{table}

\begin{figure}[t]
\centering
\includegraphics[width=0.98\linewidth]{model_comparison_with_ensemble.png}
\caption{Cross-validation RMSE comparison across all models and ensembles (lower is better).}
\label{fig:model_comparison}
\end{figure}

\noindent Key observations:
\begin{itemize}
    \item \textbf{SE-ResNet achieves the best performance} (8.6047), outperforming all gradient boosting methods by $\sim$0.15 RMSE.
    \item Among tree-based models, CatBoost achieved the best single-model performance (8.7618).
    \item Gradient boosting ensembles provide only marginal improvements over single GBM models.
    \item The neural network's advantage comes from learned feature interactions and data augmentation.
\end{itemize}

\subsection{Hyperparameter Tuning Results}
Bayesian optimization with Optuna (50 trials per model) yielded significant improvements, particularly for XGBoost. Table~\ref{tab:tuning_improvement} compares default vs.\ tuned performance.

\begin{table}[H]
\centering
\caption{Impact of Hyperparameter Tuning on Model Performance}
\label{tab:tuning_improvement}
\begin{tabular}{l c c c}
\toprule
\textbf{Model} & \textbf{Default RMSE} & \textbf{Tuned RMSE} & \textbf{Improvement} \\
\midrule
LightGBM & 8.7724 & 8.7658 & 0.0066 (0.08\%) \\
XGBoost & 8.9271 & 8.8220 & \textbf{0.1051 (1.18\%)} \\
CatBoost & 8.7618 & 8.7808 & -0.0190 (-0.22\%) \\
\bottomrule
\end{tabular}
\end{table}

\noindent Key findings from hyperparameter tuning:
\begin{itemize}
    \item \textbf{XGBoost} benefited most from tuning, with RMSE improving by 0.105 (1.18\%). The optimal configuration used deeper trees (\texttt{max\_depth}=9) with lower learning rate (0.019) and strong regularization (\texttt{reg\_lambda}=9.98).
    \item \textbf{LightGBM} showed modest improvement. Tuning favored more leaves (\texttt{num\_leaves}=84) with moderate learning rate (0.033) and higher regularization.
    \item \textbf{CatBoost} performed slightly worse after tuning, suggesting the default parameters were already near-optimal for this dataset. This highlights that automated tuning does not always guarantee improvement.
\end{itemize}

Table~\ref{tab:tuned_params} shows the optimal hyperparameters discovered.

\begin{table}[H]
\centering
\caption{Optimal Hyperparameters from Bayesian Optimization (50 trials)}
\label{tab:tuned_params}
\begin{tabular}{l l r}
\toprule
\textbf{Model} & \textbf{Parameter} & \textbf{Value} \\
\midrule
\multirow{7}{*}{LightGBM} 
    & num\_leaves & 84 \\
    & learning\_rate & 0.0334 \\
    & feature\_fraction & 0.643 \\
    & bagging\_fraction & 0.890 \\
    & min\_child\_samples & 29 \\
    & reg\_alpha & 0.043 \\
    & reg\_lambda & 9.431 \\
\midrule
\multirow{6}{*}{XGBoost}
    & max\_depth & 9 \\
    & learning\_rate & 0.0189 \\
    & subsample & 0.916 \\
    & colsample\_bytree & 0.859 \\
    & reg\_alpha & 0.090 \\
    & reg\_lambda & 9.979 \\
\midrule
\multirow{4}{*}{CatBoost}
    & depth & 6 \\
    & learning\_rate & 0.0998 \\
    & l2\_leaf\_reg & 5.133 \\
    & bagging\_temperature & 0.385 \\
\bottomrule
\end{tabular}
\end{table}

\subsection{Ensemble Optimization}
The weighted ensemble was optimized using Nelder-Mead minimization on out-of-fold predictions:
\begin{itemize}
    \item \textbf{CatBoost}: 61.98\% weight
    \item \textbf{LightGBM}: 33.16\% weight
    \item \textbf{XGBoost}: 4.86\% weight
\end{itemize}

\noindent The low weight assigned to XGBoost reflects its higher variance and slightly worse performance.

\subsection{Stacking Ensemble}
We implemented a two-level stacking architecture with Ridge regression as the meta-learner. The base models (LightGBM, XGBoost, CatBoost) generate out-of-fold predictions, which serve as meta-features for the Ridge model.

\begin{table}[H]
\centering
\caption{Stacking Meta-Learner Coefficients}
\label{tab:stacking_coef}
\begin{tabular}{l r}
\toprule
\textbf{Base Model} & \textbf{Coefficient} \\
\midrule
LightGBM & 0.334 \\
XGBoost & 0.049 \\
CatBoost & 0.620 \\
Intercept & -0.215 \\
\bottomrule
\end{tabular}
\end{table}

\noindent The stacking ensemble achieved \textbf{RMSE = 8.7567}, marginally outperforming the weighted average ensemble (8.7568). The meta-learner coefficients align closely with the optimized ensemble weights, confirming CatBoost's dominant contribution.

\subsection{Neural Network Results (SE-ResNet)}
The SE-ResNet architecture achieved the best overall performance with \textbf{RMSE = 8.6047}, representing a \textbf{1.74\% improvement} over the best gradient boosting ensemble.

\begin{table}[H]
\centering
\caption{SE-ResNet Cross-Validation Results}
\label{tab:senet_results}
\begin{tabular}{c c}
\toprule
\textbf{Fold} & \textbf{RMSE} \\
\midrule
1 & 8.612 \\
2 & 8.598 \\
3 & 8.621 \\
4 & 8.589 \\
5 & 8.603 \\
\midrule
\textbf{Mean} & \textbf{8.6047} \\
\bottomrule
\end{tabular}
\end{table}

\noindent Key factors contributing to the neural network's superior performance:
\begin{itemize}
    \item \textbf{Data augmentation}: The original dataset (not just competition data) was concatenated with each training fold, effectively increasing training data and reducing overfitting.
    \item \textbf{Learned embeddings}: Entity embeddings for categorical features allow the network to discover semantic relationships not captured by one-hot or target encoding.
    \item \textbf{SE attention}: Squeeze-and-Excitation blocks dynamically reweight features, emphasizing informative dimensions for each sample.
    \item \textbf{Residual connections}: Enable training of deeper networks without degradation.
    \item \textbf{Aggressive regularization}: Dropout (11\%), weight decay ($10^{-4}$), and early stopping prevent overfitting.
\end{itemize}

\subsection{Feature Importance}
Feature importance analysis (based on the best single model, CatBoost) revealed the top predictors:

\begin{table}[H]
\centering
\caption{Top 10 Most Important Features (CatBoost)}
\label{tab:featimp}
\begin{tabular}{l r}
\toprule
\textbf{Feature} & \textbf{Importance} \\
\midrule
formula\_score & 60.78 \\
sleep\_quality\_target\_enc & 5.99 \\
study\_method\_target\_enc & 4.95 \\
facility\_rating\_target\_enc & 4.10 \\
study\_attendance & 3.48 \\
sleep\_quality & 2.70 \\
study\_method & 2.62 \\
sleep\_quality\_freq & 2.21 \\
facility\_rating & 2.14 \\
study\_method\_freq & 1.75 \\
\bottomrule
\end{tabular}
\end{table}

\begin{figure}[t]
\centering
\includegraphics[width=0.98\linewidth]{feature_importance.png}
\caption{Feature importance for the best single model (CatBoost). The engineered \texttt{formula\_score} and category encodings dominate.}
\label{fig:feature_importance}
\end{figure}

\noindent The engineered \texttt{formula\_score} feature dominates, followed by encodings of \texttt{sleep\_quality} and \texttt{study\_method}, confirming that feature engineering significantly improved predictive power.

\subsection{SHAP Analysis}
To provide deeper insights into model behavior, we computed SHAP values~\cite{lundbergUnifiedApproachInterpreting2017} using TreeExplainer~\cite{lundbergLocalExplanationsGlobal2020} on a representative sample of 30,000 training instances.

\begin{figure}[t]
\centering
\includegraphics[width=0.98\linewidth]{shap_summary.png}
\caption{SHAP summary plot showing feature contributions to predictions. Each point represents a sample; color indicates feature value (red=high, blue=low); horizontal position shows impact on prediction.}
\label{fig:shap_summary}
\end{figure}

The SHAP analysis reveals several key insights:
\begin{itemize}
    \item \textbf{formula\_score} has the highest mean $|\text{SHAP}|$ value, confirming its dominant predictive role. Higher values consistently push predictions upward.
    \item \textbf{Target encodings} (sleep\_quality, study\_method, facility\_rating) show strong effects, with higher encoded values correlating with higher predicted scores.
    \item \textbf{study\_hours} exhibits a clear positive relationship: more study time leads to higher predicted scores.
    \item \textbf{Interaction effects}: The dependence plots reveal non-linear relationships; for instance, the effect of study hours saturates at very high values.
\end{itemize}

Unlike standard feature importance (which only measures magnitude), SHAP values provide directional information: we can see that higher study hours \textit{increase} predictions while poor sleep quality \textit{decreases} them. This interpretability is crucial for understanding and validating the model's decision-making process~\cite{lundbergExplainableMachinelearningPredictions2018}.

\subsection{Residual Analysis}
For the best model (Weighted Ensemble), residual statistics on the training set:
\begin{itemize}
    \item Mean residual: 0.0415 (near-zero bias)
    \item Residual std: 8.7567 (consistent with CV RMSE)
    \item Range: [-43.35, +48.15] (some large errors remain)
\end{itemize}

\begin{figure}[t]
\centering
\includegraphics[width=0.98\linewidth]{residual_analysis.png}
\caption{Residual analysis for the best model (optimized ensemble): predicted vs.\ actual scores and residual distribution.}
\label{fig:residuals}
\end{figure}

\section{Discussion}

\subsection{Model Performance Analysis}
The results demonstrate a clear performance hierarchy, with the neural network architecture achieving the best results:
\begin{enumerate}
    \item \textbf{SE-ResNet neural network} achieved the best performance (8.6047 RMSE), outperforming all gradient boosting methods by 1.74\%.
    \item \textbf{Gradient boosting ensembles} (stacking, weighted average) achieved $\sim$8.76 RMSE, with diminishing returns from ensemble complexity.
    \item \textbf{CatBoost} was the best single tree-based model (8.7618), slightly outperforming LightGBM due to its sophisticated categorical handling.
    \item \textbf{XGBoost} underperformed with higher variance, potentially due to suboptimal hyperparameters.
\end{enumerate}

\subsection{Why Neural Networks Outperform GBMs}
The SE-ResNet's 1.74\% improvement over gradient boosting is notable and stems from several factors:

\textbf{1. Data Augmentation:}
The most significant advantage comes from incorporating the original dataset (not just the Kaggle competition data) during training. This effectively increases the training set size and provides additional signal. GBMs were trained only on competition data, while the neural network benefited from augmented data.

\textbf{2. Learned Feature Interactions:}
While GBMs capture feature interactions through tree splits, neural networks learn continuous, differentiable interaction functions. The SE-ResNet's residual blocks with attention mechanisms can model complex, non-linear relationships that axis-aligned tree splits may miss.

\textbf{3. Entity Embeddings:}
Categorical embeddings (dimension $d=8$) allow the network to learn distributed representations where similar categories are mapped to nearby vectors. This provides richer representations than one-hot encoding or target encoding used in GBMs.

\textbf{4. Attention Mechanisms:}
Squeeze-and-Excitation blocks dynamically reweight features for each sample, effectively performing instance-specific feature selection. This adaptive behavior is not possible with static feature importance in GBMs.

\textbf{5. Regularization Strategy:}
The combination of dropout (11\%), weight decay, LayerNorm, and early stopping provides robust regularization that prevents overfitting despite the model's high capacity.

\subsection{Feature Engineering Impact}
The engineered \texttt{formula\_score} feature, based on domain knowledge from competition discussions, became the single most important predictor.
This highlights the value of incorporating domain expertise and exploring competition forums for insights.

Target encoding and interaction features (especially \texttt{study\_attendance}) contributed significantly, demonstrating that careful feature engineering can outweigh model selection in tabular competitions.

\subsection{Error Analysis}
Residual analysis revealed:
\begin{itemize}
    \item Near-zero mean residual (0.0415) indicates minimal systematic bias.
    \item Residual standard deviation (8.7567) aligns with the cross-validation RMSE, indicating consistent error magnitude.
    \item Residual range of [-43.35, +48.15] suggests difficulty predicting extreme scores.
    \item Largest errors likely occur for students with unusual combinations of features.
\end{itemize}

\subsection{Gradient Boosting vs.\ Neural Networks: Trade-offs}
While SE-ResNet achieved the best results, gradient boosting methods remain valuable:
\begin{itemize}
    \item \textbf{Training efficiency}: GBMs train in minutes; the neural network required hours.
    \item \textbf{Interpretability}: SHAP values provide clear explanations for GBMs; neural network explanations are less straightforward.
    \item \textbf{Hyperparameter sensitivity}: Neural networks require careful tuning of learning rate, architecture depth, dropout, etc.
    \item \textbf{Data requirements}: Neural networks benefit more from data augmentation; GBMs are more data-efficient.
\end{itemize}

\subsection{Limitations}
Several limitations apply to this work:
\begin{itemize}
    \item \textbf{Synthetic data}: The dataset was generated from a deep learning model, so relationships may not perfectly reflect real-world patterns. Notably, the neural network's advantage may partly stem from matching the data-generating process.
    \item \textbf{Data augmentation fairness}: The neural network used additional data (original dataset), which was not available to GBM models. A fairer comparison would train all models with the same augmented data.
    \item \textbf{Limited neural network exploration}: Only one architecture was tested; alternatives like TabNet, FT-Transformer, or NODE could yield further improvements.
    \item \textbf{Generalization}: Results are specific to this synthetic dataset and may not transfer to real student data.
\end{itemize}

\section{Conclusion}
This project addressed the Kaggle Playground Series S6E1 challenge of predicting student exam scores from tabular demographic and behavioral data.

\subsection{Summary of Contributions}
\begin{itemize}
    \item Developed a comprehensive feature engineering pipeline expanding 12 original features to 44 engineered features.
    \item Compared 7 modeling approaches from linear baselines to ensemble methods.
    \item Achieved a final CV RMSE of \textbf{8.7568} using a weighted ensemble of CatBoost (62\%), LightGBM (33\%), and XGBoost (5\%).
    \item Identified \texttt{study\_hours} and its interactions as the most predictive features.
\end{itemize}

\subsection{Key Findings}
\begin{enumerate}
    \item \textbf{Feature engineering matters}: The engineered \texttt{formula\_score} and interaction features contributed more than model selection.
    \item \textbf{Gradient boosting dominates}: CatBoost and LightGBM significantly outperformed traditional models.
    \item \textbf{Ensemble provides marginal gains}: Weighted blending improved over single models by $\sim$0.005 RMSE.
    \item \textbf{Study behavior is key}: Study hours and class attendance are the strongest predictors of exam performance.
\end{enumerate}

\subsection{Future Work}
Potential improvements include:
\begin{itemize}
    \item Systematic hyperparameter optimization using Optuna or Bayesian methods
    \item Neural network approaches (TabNet, deep embeddings)
    \item Stacking ensembles with meta-learners
    \item Incorporating the original (non-synthetic) dataset for additional training signal
\end{itemize}


% -------------------- References --------------------
\bibliographystyle{ieeetr}
\bibliography{references}

% -------------------- Appendix (optional) --------------------
\appendix
\section{Reproducibility and Implementation Details}
\label{app:repro}

All experiments used a fixed random seed (SEED=42) for dataset splitting and
cross-validation. Implementation was done in Python using:
\begin{itemize}
    \item pandas, numpy for data manipulation
    \item scikit-learn for preprocessing and baseline models
    \item LightGBM, XGBoost, CatBoost for gradient boosting
    \item scipy for ensemble weight optimization
\end{itemize}

\subsection{Final Model Hyperparameters}
\label{app:final_hparams}
Tables~\ref{tab:hparams_lgb}, \ref{tab:hparams_cat}, and \ref{tab:hparams_xgb}
report the final hyperparameter configurations used for training the gradient
boosting models in all experiments.

\begin{table}[H]
\centering
\caption{LightGBM final hyperparameters}
\label{tab:hparams_lgb}
\begin{tabular}{l c}
\toprule
\textbf{Parameter} & \textbf{Value} \\
\midrule
learning\_rate & 0.05 \\
num\_leaves & 31 \\
feature\_fraction & 0.9 \\
bagging\_fraction & 0.8 \\
min\_child\_samples & 20 \\
reg\_alpha & 0.1 \\
reg\_lambda & 0.1 \\
early\_stopping\_rounds & 100 \\
\bottomrule
\end{tabular}
\end{table}

\begin{table}[H]
\centering
\caption{CatBoost final hyperparameters}
\label{tab:hparams_cat}
\begin{tabular}{l c}
\toprule
\textbf{Parameter} & \textbf{Value} \\
\midrule
learning\_rate & 0.05 \\
depth & 6 \\
l2\_leaf\_reg & 3 \\
iterations & 10000 \\
early\_stopping\_rounds & 100 \\
\bottomrule
\end{tabular}
\end{table}

\begin{table}[H]
\centering
\caption{XGBoost final hyperparameters}
\label{tab:hparams_xgb}
\begin{tabular}{l c}
\toprule
\textbf{Parameter} & \textbf{Value} \\
\midrule
learning\_rate & 0.05 \\
max\_depth & 6 \\
subsample & 0.8 \\
colsample\_bytree & 0.9 \\
reg\_alpha & 0.1 \\
reg\_lambda & 0.1 \\
tree\_method & hist \\
early\_stopping\_rounds & 100 \\
\bottomrule
\end{tabular}
\end{table}

\subsection{Optuna Hyperparameter Search Spaces}
\label{app:search_spaces}
The hyperparameter ranges explored during Bayesian optimization with Optuna are
summarized in Table~\ref{tab:search_space}. These search spaces define the bounds
within which Optuna sampled configurations during the tuning process described
in Section~IV.

\begin{table}[H]
\centering
\renewcommand{\arraystretch}{1.1}
\caption{Optuna search spaces used for Bayesian hyperparameter optimization.}
\label{tab:search_space}
\begin{tabular}{l l l}
\hline
\textbf{Model} & \textbf{Hyperparameter} & \textbf{Range} \\
\hline
\multirow{7}{*}{LightGBM}
  & \texttt{num\_leaves}         & $[20, 100]$ \\
  & \texttt{learning\_rate}      & $[0.01, 0.1]$ \\
  & \texttt{feature\_fraction}   & $[0.6, 1.0]$ \\
  & \texttt{bagging\_fraction}   & $[0.6, 1.0]$ \\
  & \texttt{min\_child\_samples} & $[5, 50]$ \\
  & \texttt{reg\_alpha}          & $[10^{-3}, 10]$ \\
  & \texttt{reg\_lambda}         & $[10^{-3}, 10]$ \\
\hline
\multirow{6}{*}{XGBoost}
  & \texttt{max\_depth}          & $[3, 10]$ \\
  & \texttt{learning\_rate}      & $[0.01, 0.1]$ \\
  & \texttt{subsample}           & $[0.6, 1.0]$ \\
  & \texttt{colsample\_bytree}   & $[0.6, 1.0]$ \\
  & \texttt{reg\_alpha}          & $[10^{-3}, 10]$ \\
  & \texttt{reg\_lambda}         & $[10^{-3}, 10]$ \\
\hline
\multirow{4}{*}{CatBoost}
  & \texttt{depth}               & $[4, 10]$ \\
  & \texttt{learning\_rate}      & $[0.01, 0.1]$ \\
  & \texttt{l2\_leaf\_reg}       & $[1, 10]$ \\
  & \texttt{bagging\_temperature}& $[0.0, 1.0]$ \\
\hline
\end{tabular}
\end{table}

\subsection{Ensemble Weights}
\label{app:ensemble}
The final weighted ensemble used the following optimized weights:
\begin{itemize}
    \item CatBoost: 61.98\%
    \item LightGBM: 33.16\%
    \item XGBoost: 4.86\%
\end{itemize}


\end{document}

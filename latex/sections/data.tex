\section{Dataset Description and Preprocessing}
We use the dataset provided in the Kaggle Playground Series S6E1 benchmark, which consists of synthetically generated student records derived from a model trained on real student performance data.
The training split contains 630{,}000 samples with the target label \texttt{exam\_score}, and the test split contains 270{,}000 samples used for final prediction. Figure~\ref{fig:target_distribution} visualizes the distribution of \texttt{exam\_score}, which is approximately symmetric and concentrated around mid-range values.
\begin{figure}[t]
\centering
\includegraphics[width=0.98\linewidth]{target_distribution.png}
\caption{Distribution of the target variable \texttt{exam\_score} (histogram and box plot).}
\label{fig:target_distribution}
\end{figure}
\subsection{Feature Overview}


\begin{table}[h]
\centering
\renewcommand{\arraystretch}{1.1}
\caption{Overview of input features.}
\label{tab:feature_overview}
\begin{tabular}{l l l}
\hline
\textbf{Type} & \textbf{Feature} & \textbf{Range / Categories} \\
\hline
\multirow{4}{*}{Numerical} 
  & \texttt{age}              & 17--24 \\
  & \texttt{study\_hours}     & 0.08--7.91 \\
  & \texttt{class\_attendance}& 40.6--99.4\% \\
  & \texttt{sleep\_hours}     & 4.1--9.9 \\
\hline
\multirow{7}{*}{Categorical}
  & \texttt{gender}           & male, female \\
  & \texttt{course}           & B.Tech, B.Sc, etc. \\
  & \texttt{internet\_access} & yes, no \\
  & \texttt{sleep\_quality}   & good, average, poor \\
  & \texttt{study\_method}    & category labels \\
  & \texttt{facility\_rating} & category labels \\
  & \texttt{exam\_difficulty} & category labels \\
\hline
\end{tabular}
\end{table}

Table~\ref{tab:feature_overview} summarizes the numerical and categorical input variables.


\subsection{Preprocessing}

We verified that the dataset contains no missing values in either the training or test split.
Categorical variables were converted into numerical representations using label encoding, and we additionally applied frequency encoding to capture category prevalence.
To incorporate target-dependent information while avoiding leakage, we implemented cross-validation-safe target encoding, where category statistics are computed using only the training folds and then applied to the corresponding validation fold.
All preprocessing steps were applied consistently across training and test data to ensure compatibility with both tree-based models and neural architectures.

\subsection{Exploratory Data Analysis (EDA)}
In this subsection, we perform exploratory data analysis to identify the most informative numerical predictors of the target \texttt{exam\_score} and to motivate subsequent modeling choices.
As summarized in the Pearson correlation matrix (Figure~\ref{fig:corr_matrix}), \texttt{study\_hours} exhibits the strongest linear association with the target ($r=0.762$), making it the most relevant numerical feature.
\texttt{class\_attendance} shows a moderate positive correlation with exam performance ($r=0.361$), while \texttt{sleep\_hours} has a weaker but still positive relationship ($r=0.167$).
In contrast, \texttt{age} has negligible correlation with \texttt{exam\_score} ($r=0.011$), suggesting limited predictive value as a standalone input.

These findings are consistent with the scatter plots in Figure~\ref{fig:scatter_num_target}.
\texttt{study\_hours} displays a clear upward trend with relatively tight alignment around the fitted line, whereas \texttt{class\_attendance} shows a weaker trend with substantially higher dispersion.
\texttt{sleep\_hours} exhibits only a mild positive trend and considerable variance across the score range.
Finally, \texttt{age} appears as discrete vertical bands with no meaningful slope, reinforcing the absence of a strong linear relationship.

Beyond feature-target relationships, Figure~\ref{fig:corr_matrix} also indicates low pairwise correlations among the numerical predictors, suggesting limited multicollinearity within the numerical subset.
Overall, the EDA identifies \texttt{study\_hours}, \texttt{class\_attendance}, and \texttt{sleep\_hours} as the three most relevant numerical features for predicting exam scores, motivating the use of interaction features and nonlinear models in later sections.

\begin{figure}[t]
\centering
\includegraphics[width=0.98\linewidth]{correlation_matrix.png}
\caption{Pearson correlation matrix of numerical features and the target \texttt{exam\_score}.}
\label{fig:corr_matrix}
\end{figure}

\begin{figure*}[t]
\centering
\includegraphics[width=0.95\textwidth]{scatter_numerical_vs_target.png}
\caption{Scatter plots of numerical features versus \texttt{exam\_score} with linear trend lines (sampled for visualization).}
\label{fig:scatter_num_target}
\end{figure*}

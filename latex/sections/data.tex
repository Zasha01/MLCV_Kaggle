\section{Dataset Description and Preprocessing}
The dataset originates from the Kaggle Playground Series S6E1 competition, consisting of synthetically generated student data.
The training set contains 630,000 samples with the target label \texttt{exam\_score}, while the test set contains 270,000 samples requiring predictions.

\subsection{Feature Overview}
The dataset contains 12 original features (excluding \texttt{id}), split into:
\begin{itemize}
    \item \textbf{Numerical (4)}: age (17--24), study\_hours (0.08--7.91), class\_attendance (40.6--99.4\%), sleep\_hours (4.1--9.9)
    \item \textbf{Categorical (7)}: gender, course (B.Tech, B.Sc, etc.), internet\_access, sleep\_quality (good/average/poor), study\_method, facility\_rating, exam\_difficulty
\end{itemize}

\noindent The target variable \texttt{exam\_score} has the following distribution:
\begin{itemize}
    \item Mean: 62.51, Standard Deviation: 18.92
    \item Range: 19.60 -- 100.00
    \item Skewness: -0.048 (approximately symmetric)
\end{itemize}

\begin{figure}[t]
\centering
\includegraphics[width=0.98\linewidth]{target_distribution.png}
\caption{Distribution of the target variable \texttt{exam\_score} (histogram and box plot).}
\label{fig:target_distribution}
\end{figure}

\subsection{Data Cleaning}
We performed the following preprocessing steps:
\begin{itemize}
    \item Verified \textbf{no missing values} in both training and test sets.
    \item Applied label encoding for categorical variables.
    \item Implemented CV-safe target encoding to prevent data leakage.
    \item Added frequency encoding for categorical features.
\end{itemize}

\subsection{Exploratory Data Analysis (EDA)}
In this subsection, we perform exploratory data analysis to identify the most informative numerical predictors of the target \texttt{exam\_score} and to motivate subsequent modeling choices.
As summarized in the Pearson correlation matrix (Figure~\ref{fig:corr_matrix}), \texttt{study\_hours} exhibits the strongest linear association with the target ($r=0.762$), making it the most relevant numerical feature.
\texttt{class\_attendance} shows a moderate positive correlation with exam performance ($r=0.361$), while \texttt{sleep\_hours} has a weaker but still positive relationship ($r=0.167$).
In contrast, \texttt{age} has negligible correlation with \texttt{exam\_score} ($r=0.011$), suggesting limited predictive value as a standalone input.

These findings are consistent with the scatter plots in Figure~\ref{fig:scatter_num_target}.
\texttt{study\_hours} displays a clear upward trend with relatively tight alignment around the fitted line, whereas \texttt{class\_attendance} shows a weaker trend with substantially higher dispersion.
\texttt{sleep\_hours} exhibits only a mild positive trend and considerable variance across the score range.
Finally, \texttt{age} appears as discrete vertical bands with no meaningful slope, reinforcing the absence of a strong linear relationship.

Beyond feature-target relationships, Figure~\ref{fig:corr_matrix} also indicates low pairwise correlations among the numerical predictors, suggesting limited multicollinearity within the numerical subset.
Overall, the EDA identifies \texttt{study\_hours}, \texttt{class\_attendance}, and \texttt{sleep\_hours} as the three most relevant numerical features for predicting exam scores, motivating the use of interaction features and nonlinear models in later sections.

\begin{figure}[t]
\centering
\includegraphics[width=0.98\linewidth]{correlation_matrix.png}
\caption{Pearson correlation matrix of numerical features and the target \texttt{exam\_score}.}
\label{fig:corr_matrix}
\end{figure}

\begin{figure*}[t]
\centering
\includegraphics[width=0.95\textwidth]{scatter_numerical_vs_target.png}
\caption{Scatter plots of numerical features versus \texttt{exam\_score} with linear trend lines (sampled for visualization).}
\label{fig:scatter_num_target}
\end{figure*}

\section{Dataset Description and Preprocessing}
We use the dataset provided in the Kaggle Playground Series S6E1 benchmark, which
consists of synthetically generated student records derived from a model trained
on real student performance data. The training split contains 630{,}000 samples
with the target variable \texttt{exam\_score}, while the test split contains
270{,}000 samples used for final prediction. Figure~\ref{fig:target_distribution}
visualizes the distribution of \texttt{exam\_score}, which is approximately
symmetric and concentrated around mid-range values.

\begin{figure}[t]
\centering
\includegraphics[width=0.98\linewidth]{target_distribution.png}
\caption{Distribution of the target variable \texttt{exam\_score} (histogram and box plot).}
\label{fig:target_distribution}
\end{figure}
\subsection{Feature Overview}
Table~\ref{tab:feature_overview}  summarizes the numerical and categorical input
variables.

\begin{table}[h]
\centering
\renewcommand{\arraystretch}{1.1}
\caption{Overview of input features.}
\label{tab:feature_overview}
\begin{tabular}{l l l}
\hline
\textbf{Type} & \textbf{Feature} & \textbf{Range / Categories} \\
\hline
\multirow{4}{*}{Numerical} 
  & \texttt{age}               & 17--24 \\
  & \texttt{study\_hours}      & 0.08--7.91 \\
  & \texttt{class\_attendance} & 40.6--99.4\% \\
  & \texttt{sleep\_hours}      & 4.1--9.9 \\
\hline
\multirow{7}{*}{Categorical}
  & \texttt{gender}            & male, female \\
  & \texttt{course}            & B.Tech, B.Sc, etc. \\
  & \texttt{internet\_access}  & yes, no \\
  & \texttt{sleep\_quality}    & good, average, poor \\
  & \texttt{study\_method}     & category labels \\
  & \texttt{facility\_rating}  & category labels \\
  & \texttt{exam\_difficulty}  & category labels \\
\hline
\end{tabular}
\end{table}

\subsection{Preprocessing}
We verified that the dataset contains no missing values in
either the training or test split. Categorical variables were
converted into numerical representations using label encoding, 
and we additionally applied frequency encoding to capture 
category prevalence. To incorporate target-dependent information
while avoiding leakage, we implemented cross-validation-safe
target encoding, where category statistics are computed using
only the training folds and then applied to the corresponding
validation fold. All preprocessing steps were applied consistently 
across training and test data to ensure compatibility with
both tree-based models and neural architectures.

\subsection{Exploratory Data Analysis}
Exploratory analysis was conducted to assess relationships between numerical
features and the target variable. As shown in
Figure~\ref{fig:corr_matrix}, \texttt{study\_hours} exhibits the strongest linear
association with \texttt{exam\_score} ($r=0.762$), followed by
\texttt{class\_attendance} ($r=0.361$) and \texttt{sleep\_hours} ($r=0.167$).
In contrast, \texttt{age} shows negligible correlation ($r=0.011$).

These findings are consistent with the scatter plots in
Figure~\ref{fig:scatter_num_target}, which indicate a clear positive relationship
between study time and exam performance, weaker but positive effects for
attendance and sleep duration, and no meaningful trend for age. Low pairwise
correlations among numerical features suggest limited multicollinearity.
Overall, the analysis identifies \texttt{study\_hours},
\texttt{class\_attendance}, and \texttt{sleep\_hours} as the most informative
numerical predictors, motivating the use of interaction features and nonlinear
models in later sections.

\begin{figure}[t]
\centering
\includegraphics[width=0.98\linewidth]{correlation_matrix.png}
\caption{Pearson correlation matrix of numerical features and the target \texttt{exam\_score}.}
\label{fig:corr_matrix}
\end{figure}

\begin{figure}[t]
\centering
\includegraphics[width=0.95\linewidth]{scatter_numerical_vs_target.png}
\caption{Scatter plots of numerical features versus \texttt{exam\_score} with linear trend lines (sampled for visualization).}
\label{fig:scatter_num_target}
\end{figure}

\section{Dataset Description and Preprocessing}
The dataset originates from the Kaggle Playground Series S6E1 competition, consisting of synthetically generated student data.
The training set contains 630,000 samples with the target label \texttt{exam\_score}, while the test set contains 270,000 samples requiring predictions.

\subsection{Feature Overview}
The dataset contains 12 original features (excluding \texttt{id}), split into:
\begin{itemize}
    \item \textbf{Numerical (4)}: age (17--24), study\_hours (0.08--7.91), class\_attendance (40.6--99.4\%), sleep\_hours (4.1--9.9)
    \item \textbf{Categorical (7)}: gender, course (B.Tech, B.Sc, etc.), internet\_access, sleep\_quality (good/average/poor), study\_method, facility\_rating, exam\_difficulty
\end{itemize}

\noindent The target variable \texttt{exam\_score} has the following distribution:
\begin{itemize}
    \item Mean: 62.51, Standard Deviation: 18.92
    \item Range: 19.60 -- 100.00
    \item Skewness: -0.048 (approximately symmetric)
\end{itemize}

\begin{figure}[t]
\centering
\includegraphics[width=0.98\linewidth]{target_distribution.png}
\caption{Distribution of the target variable \texttt{exam\_score} (histogram and box plot).}
\label{fig:target_distribution}
\end{figure}

\subsection{Data Cleaning}
We performed the following preprocessing steps:
\begin{itemize}
    \item Verified \textbf{no missing values} in both training and test sets.
    \item Applied label encoding for categorical variables.
    \item Implemented CV-safe target encoding to prevent data leakage.
    \item Added frequency encoding for categorical features.
\end{itemize}

\subsection{Exploratory Data Analysis (EDA)}
EDA revealed important relationships between features and exam scores:
\begin{itemize}
    \item \textbf{study\_hours} shows the strongest correlation (r = 0.762) with exam score.
    \item \textbf{class\_attendance} has moderate positive correlation (r = 0.361).
    \item \textbf{sleep\_hours} exhibits weak positive correlation (r = 0.167).
    \item \textbf{age} shows negligible correlation (r = 0.011).
\end{itemize}

\noindent Figure~\ref{fig:corr_matrix} summarizes linear correlations among numerical variables, while Figures~\ref{fig:scatter_num_target} and \ref{fig:cat_boxplots} illustrate feature-target relationships for numerical and categorical variables.

\begin{figure}[t]
\centering
\includegraphics[width=0.98\linewidth]{correlation_matrix.png}
\caption{Correlation matrix of numerical features and the target.}
\label{fig:corr_matrix}
\end{figure}

\begin{figure*}[t]
\centering
\includegraphics[width=0.95\textwidth]{scatter_numerical_vs_target.png}
\caption{Scatter plots of numerical features versus \texttt{exam\_score} with linear trend lines (sampled for visualization).}
\label{fig:scatter_num_target}
\end{figure*}

\begin{figure*}[t]
\centering
\includegraphics[width=0.95\textwidth]{categorical_boxplots.png}
\caption{Score distributions across categorical features. Categories with higher \texttt{sleep\_quality}, \texttt{study\_method}, and \texttt{facility\_rating} tend to show higher median scores.}
\label{fig:cat_boxplots}
\end{figure*}

\noindent Categorical variables like \texttt{sleep\_quality}, \texttt{study\_method}, and \texttt{facility\_rating} showed meaningful differences in score distributions across categories.
